% Preamble
\documentclass[../Data_Science.tex]{subfiles}

% Packages
% Document

\begin{document}
\section{Ch01. What is Data Science?}\label{sec:Ch01. What is Data Science?}
% bold font

DataScience is \tb{Learning about the world from data using computation.}
\begin{itemize}
    \item Exploration
    \item \begin{itemize}
              \item Identifying patterns in data
              \item Uses visulization
    \end{itemize}
    \item Inference
    \item \begin{itemize}
              \item Using data to draw reliabel conclusions about the world
              \item Uses statics
    \end{itemize}
    \item Prediction
    \item \begin{itemize}
              \item Making informed guesse about unobserved data
              \item Uses Machine learning
    \end{itemize}
    \item Statistics
    \item  [] 불완전한 정보를 튼튼한 결론으로 만들어준다.
    \item Computing
    \item [] 분석 기술들을 우리에게 실행시켜준다.
\end{itemize}
\subsection{Why Data Science?}\label{subsec:Why Data Science?}
대부분 중요한 결정들은 불확실한 결과와 부분적인 정보들로만 만들어지기 때문이다. 즉 빈약한 정보들로 완강한 결론을 만들어 내야 한다.

\subsection{Criticla Thinking and Data Science}\label{subsec:Criticla Thinking and Data Science}
\subsubsection{Critical Thinking}\label{subsubsec:Critical Thinking}
엄격한 교육읜 산물이며 데이터가 해당 비판을 지지할 때 가장 효과적이다. 세상의 어떤 면에 대한 비판적 분석은 귀납적 추론을 포함한다.
\end{document}